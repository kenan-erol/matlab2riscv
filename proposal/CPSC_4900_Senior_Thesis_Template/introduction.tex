Essentially, the introduction is a summary of the paper. A great introduction is one that makes the reader excited to read the rest of the paper. A good introduction encourages readers to read your work with interest and prepares them to understand it better. 

Remember the three-pass reading that we learned in Lecture 2 of the class, CPSC 4900 Senior Project? 
When you write a paper, you can expect most reviewers (and readers) to make only one pass over it. (i.e. 5 min, max 10 min.) Then, decide whether they will read more or put it aside. 


A strong introduction does the following in this general order:
\begin{itemize}
    \item Motivates why the subject of the research is important
    \item States the research question
    \item Discusses where the paper fits in existing literature
    \item Describes the contribution of the paper
    \item States the research methods
    \item States the main results
\end{itemize}


In this template, section titles are provided as suggestions, and you are more than welcome to change them, except for Abstract, Introduction, Background, Related Work, and Conclusion, which are must-have. 

Please share the document with your advisor with edit previllage. Comments can be provided \blue { this way, and comments can be removed when addressed.}