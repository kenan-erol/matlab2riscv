Organize the result section according to major topics. 

\section{subsection}
The reader will scan through the section headers at the first pass. Subsection headings help organizations and help the reader find the parts of interest.

Make section headings as specific and information-rich as possible. Make sure to interpret the results. Summarize the collection of experimental results as clearly and as economically as possible. 

\section{How you should prepare figures}
\begin{figure}[h]
    \centering
    \includegraphics[width=0.3\textwidth]{figs/QR_PosterSessionDirectory.png}
    \vskip -0.2in
    \caption{QR Code for Poster Session Program, Spring 2024.}
    \label{fig:qr_poster}
\end{figure}
The figure and table captions should contain enough context so that a reader can understand the content of the figure or table without having to refer to the text.

Any labels or uncommon abbreviations need to be explained in the figure or table caption.

\section{In the Results section.. }

\begin{itemize}
\item Organize this section according to major topics. 
\item Subheadings to make the organization clear and to help the reader \item scan through the text to find the parts of interest.
\item Make these section headings as specific and information-rich as possible but try to cover the major common points.
\item Interpret your results. (e.g., How do your results compare to related studies? Do your results accord with theory or are they surprising in some way? What are the underlying mechanisms that may explain what you found?)
\item Summarize the collection of experimental results as clearly and as economically as possible. 
\item The figure and table captions should contain enough context so that a reader can understand the content of the figure or table without having to refer to the text.
\item Any labels or uncommon abbreviations need to be explained in the figure or table caption.
\end{itemize}


